\subsection*{Иродов 2.32}

\setcounter{equation}{0}

\begin{abstract}
Система состоит из шара радиуса $R$, заряженого сферически-симметрично, и окружающей среды, заполненной зарядом с
объемной плотностью $\rho = \alpha/r$, где $\alpha$ — постоянная, $r$ — расстояние от центра шара. Пренебрегая влиянием вещества, найти заряд шара, при котором модуль напряженности электрического
поля вне шара не зависит от $r$. Чему равна эта напряженность?
\end{abstract}

\noindent \hrulefill
\\
Заряд в окрестностях шара в точке с радиусом r > R:  

$$q'= \displaystyle \int_{R}^{r} \rho \times dV = \displaystyle \int_{R}^{r} \rho \times 4 \pi r^2 dr = 4 \pi \alpha\displaystyle \int_{R}^{r} r .dr = 2 \pi \alpha (r^2 - R^2)  $$

Согласно теореме Гаусса: 
$$\Phi_E = E \times 4 \pi r^2 = \frac{q + q'}{\epsilon_0}$$
Напряженность электрического поля вокруг сферы радиусом r: 
$$ \xrightarrow{} E = \frac{1}{4 \pi \epsilon_0} \times \frac{q + q'}{r^2} = k \times \frac{q + 2 \pi \alpha(r^2 - R^2)}{r^2} = k \times (2 \pi \alpha + \frac{q - 2 \pi \alpha R^2}{r^2})$$
модуль напряженности электрического
поля вне шара не зависит от $r$,  то есть:
$$ q = 2 \pi \alpha R^2 \xrightarrow[]{} E = k \times 2 \pi \alpha = \frac{\alpha}{2 \epsilon_0}$$



\textbf{Ответ:}
$$ q = 2 \pi \alpha R^2 $$
$$E = k \times 2 \pi \alpha = \frac{\alpha}{2 \epsilon_0}$$


