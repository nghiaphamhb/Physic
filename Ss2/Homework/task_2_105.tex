\subsection*{Иродов 2.105}

\setcounter{equation}{0}

\begin{abstract}
Пластинка толщины $h$ из однородного статически поляризованного диэлектрика находится внутри плоского конденсатора, обкладки которого соединены между собой проводником. Поляризованность диэлектрика равна $P$ (рис. 2.16). Расстояние между обкладками конденсатора $d$. Найти векторы
$E$ и $D$ внутри и вне пластины.
\end{abstract}

\noindent \hrulefill
\\
\begin{wrapfigure}[6]{r}{0.30\textwidth}
	\raisebox{0pt}[\dimexpr\height-2\baselineskip\relax]{
	\includegraphics[width=0.30\textwidth]{pics/2_105.jpeg}}
\end{wrapfigure}

$E_1$ и $E_2$ - это внутреннее и внешнее электрические поля диэлектрика соответственно.

Две обкладки конденсатора соединены, поэтому разность потенциалов между ними равна 0.

$$E_1.(d-h) + E_2.h = 0 \qquad (1)$$

Электрические смещения в системе должны быть равны, чтобы удовлетворять условию постоянства электрического поля через диэлектрическую пластину.

$$D = \varepsilon_0 . E_1 = \varepsilon_0 . E_2 + P \qquad (2)$$

Из $(1), (2):$

$$E_1 = - \frac{E_2.h}{d-h} = E_2 + \frac{P}{\varepsilon_0} \xrightarrow[]{} E_2 = -\frac{P(d-h)}{d \varepsilon_0}$$

$$E_1 = \frac{Ph}{d \varepsilon_0}; \qquad D = \varepsilon_0 . E_1 = \frac{Ph}{d}$$

\textbf{Ответ:}

$$E_1 = \frac{Ph}{d \varepsilon_0}; \qquad E_2 = -\frac{P(d-h)}{d \varepsilon_0}; \qquad D =  \frac{Ph}{d}$$







