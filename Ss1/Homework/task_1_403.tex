\subsection*{Иродов 1.403}

\setcounter{equation}{0}

\begin{abstract}
В K-системе отсчёта мюон, движущийся со скоростью $v = 0.990 \ c$, пролетел от места своего рождения до точки распада расстояние $l = 3.0$ км. Определить:
\begin{enumerate}
    \item Собственное время жизни этого мезона;
    \item Расстояние, которое пролетел мюон в K-системе с "его точки зрения".
\end{enumerate}
\end{abstract}

\noindent\hrulefill

1. Время жизни мюона в K-системе:
\begin{equation*}
    \Delta t = \frac{l}{v} = \frac{3000}{0.990 \cdot 3 \cdot 10^8} 
    = 1.01 \cdot 10^{-5} \ \text{с} = 10.1 \ \text{мкс}.
\end{equation*}

Формула для замедления времени:
\begin{equation*}
    \Delta t_0 = \Delta t \cdot \sqrt{1 - \frac{v^2}{c^2}} =  1.42 \cdot 10^{-6} \quad c.
\end{equation*}

2. Теперь определим расстояние в K-системе с точки зрения мюона::
\begin{equation*}
    l_0 = v \cdot \Delta t_0.
\end{equation*}

Итоговые формулы:
\begin{equation*}
    l_0 = v \cdot \frac{l}{v} \cdot \sqrt{1 - \frac{v^2}{c^2}}
\end{equation*}
Подставим значения:
\begin{align*}
    l_0 &= 0.990 \cdot c \cdot 1.42 \cdot 10^{-6} \\
    &= 0.990 \cdot 3 \cdot 10^8 \cdot 1.42 \cdot 10^{-6} \\
    &\approx 420 \ \text{м}.
\end{align*}
\

ОТВЕТ: 
\begin{enumerate}
    \item Собственное время жизни мезона: $\Delta t_0 \approx 1.42 \ \text{мкс}$.
    \item Расстояние, пройденное мезоном: $l_0 \approx 420 \ \text{м}$.
\end{enumerate}