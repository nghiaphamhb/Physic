\subsection*{Иродов 1.396}

\setcounter{equation}{0}

\begin{abstract}
Стержень движется в продольном направлении с постоянной скоростью \( v \) относительно инерциальной \( K \)-системы отсчета. 
При каком значении \( v \) длина стержня в этой системе отсчета будет на \( \eta = 0.5\% \) меньше его собственной длины?
\end{abstract}

\noindent\hrulefill

Формула Лоренцева сокращения длины задаётся как:
\begin{equation*}
    l = l_0 \sqrt{1 - \left( \frac{v}{c} \right)^2}
\end{equation*}

По условию:
\begin{equation*}
    l = l_0 (1 - \eta)
\end{equation*}

Приравниваем:
\begin{align*}
    \sqrt{1 - \left( \frac{v}{c} \right)^2} &= 1 - \eta \\
    \Rightarrow 1 - \left( \frac{v}{c} \right)^2 &= (1 - \eta)^2 \\
    \Rightarrow 1 -\left( \frac{v}{c} \right)^2 &= 1 -2\eta + \eta^2 \\
    \Rightarrow \left( \frac{v}{c} \right)^2 &= \eta (2 - \eta) \\
    \Rightarrow v &= c \sqrt{\eta (2 - \eta)}.
\end{align*}


При \( \eta = 0.1 \), получаем:
\begin{align*}
    v &= 0.1c.
\end{align*}

Таким образом, скорость вычисляется как \( v = 0.1c \).


ОТВЕТ: \( v = 0.1c \).