\subsection*{Иродов 1.397}

\setcounter{equation}{0}

Имеется прямоугольный треугольник, у которого катет \( a = 5,00 \, \text{м} \) и угол между этим катетом и гипотенузой \( \alpha = 30^\circ \). Найти в системе отсчета \( K' \), движущейся относительно этого треугольника со скоростью \( v = 0.866c \) вдоль катета \( a \):
\begin{itemize}
    \item[а)] соответствующее значение угла \( \alpha' \);
    \item[б)] длину \( l' \) гипотенузы и её отношение к собственной длине.
\end{itemize}

\noindent\hrulefill

а) 
В системе отсчета \( K' \), движущейся вдоль катета \( a \), катет \( a \) сокращается по формуле Лоренцева сокращения длины:
\[
a' = a \sqrt{1 - \beta^2}, \quad \text{где } \beta = \frac{v}{c}.
\]


Гипотенуза \( l \) и другой катет \( b \) остаются неизменными, так как движение происходит только вдоль катета \( a \).
Соотношение тангенсов углов \( \alpha \) и \( \alpha' \) определяется следующим образом:
\[
\tan(\alpha') = \frac{b}{a'}, \quad \tan(\alpha) = \frac{b}{a}.
\]

\[
\rightarrow \tan(\alpha') = \tan(\alpha) \cdot \frac{a}{a'} =   \frac{\tan(\alpha)}{\sqrt{1 - \beta^2}}.
\]

\[
\rightarrow  \alpha' \approx 49.4^\circ.
\]
б)
Собственная длина гипотенузы \( l \) определяется как:
\[
l = \sqrt{a^2 + b^2}.
\]

В системе отсчета \( K' \), гипотенуза \( l' \) также сокращается по формуле:
\[
l' = \sqrt{a'^2 + b^2} = \sqrt{a^2(1 - \beta^2) + b^2} = \sqrt{l^2 - a^2\beta^2}
\]
Отношение длины гипотенузы к собственной длине:

\[
\frac{l'}{l} = \sqrt{1 - \frac{a^2 \beta^2}{l^2}} = \sqrt{1 - (\cos \alpha)^2 \beta^2} \approx 0,66.
\]

ОТВЕТ: 
\begin{itemize}
    \item[а)] Новый угол \( \alpha' \approx 49.4^\circ \).
    \item[б)] Отношение длины гипотенузы\( \frac{l'}{l} \approx 0,66 \).
\end{itemize}
