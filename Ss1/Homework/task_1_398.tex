\subsection*{Иродов 1.398}

\setcounter{equation}{0}
\begin{abstract}
Найти собственную длину стержня, если в K-системе отсчета его скорость \( v = \frac{c}{2} \), длина \( l = 1,00 \, \text{м} \) и угол между ним и направлением движения \( \theta = 45^\circ \).
\end{abstract}

\noindent\hrulefill

В покоящейся системе отсчета координат: \( \theta_{0}\) - угол между стержнем и направлением движения; \( \ell_0 \) — собственная длина стержня.
\[
\beta = \frac{v}{c} = \frac{1}{2}
\]
Наблюдая за явлением, только сжимается компонента длины в направлении движения:

\[
\ell \sin \theta = \ell_0 \sin \theta \quad \xrightarrow{} \quad \left( \ell_0 \cos \theta_0 \right)^2 = \left( \ell_0 \right)^2 - \left( \ell \sin \theta \right)^2   \quad (1)
\]

Длина \(\ell \) считается как:

\[
\ell = \sqrt{\ell_0^2 \cos^2 \theta_0 \left( 1 - \beta^2 \right) + \ell_0^2 \sin^2 \theta_0}
\]
Раскроем:

\[
l^2 = l_0^2 \cos^2 \theta_0  - l_0^2 \cos^2 \theta_0\beta^2 + l_0^2 \sin^2 \theta_0
\]

\[
l^2 = l_0^2  - l_0^2 \cos^2 \theta_0\beta^2 \qquad \qquad  \qquad (2)
\]

Из \((1)\) и \((2)\):
\[
l^2 = l_0^2  - [\left( \ell_0 \right)^2 - \left( \ell \sin \theta \right)^2]\beta^2
\]

\[
l^2 (1 - \beta^2 \sin^2 \theta) = l_0^2(1-\beta^2) 
\]

\[
\xrightarrow{} l_0 = \frac{l \sqrt{1 - \beta^2 \sin^2 \theta}}{\sqrt{1 - \beta^2}}
\]
Подставляем:

\[
\ell_0 = \frac{1 \cdot \sqrt{1 - \left( \frac{1}{2} \right)^2 \sin^2 45^\circ}}{\sqrt{1 - \left( \frac{1}{2} \right)^2}} = 1,08 \, \text{м}.
\]



ОТВЕТ: 

\(\ell_0 =  1,08 \, \text{м}.\)
