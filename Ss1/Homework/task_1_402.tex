\subsection*{Иродов 1.402}

\setcounter{equation}{0}

\begin{abstract}
Собственное время жизни некоторой нестабильной частицы $\Delta t_0 = 10$ нс. Найти путь, который пролетит эта частица до распада в лабораторной системе отсчёта, где её время жизни $\Delta t = 20$ нс.
\end{abstract}

\noindent\hrulefill

\textbf{Формула для замедления времени:}
\begin{equation*}
\Delta t = \frac{\Delta t_0}{\sqrt{1 - (v/c)^2}}.
\end{equation*}

Отсюда находим скорость $v$ частицы:
\begin{align*}
1 - (v/c)^2 &= \left(\frac{\Delta t_0}{\Delta t}\right)^2 \\
\Longleftrightarrow  (v/c)^2 &= 1 - \left(\frac{\Delta t_0}{\Delta t}\right)^2 \\
\Longrightarrow v &= c \cdot \sqrt{1 - \left(\frac{\Delta t_0}{\Delta t}\right)^2}.
\end{align*}

Путь $S$, пройденный частицей за время $\Delta t$:
\begin{align*}
S &= v \cdot \Delta t \\
  &= c \cdot \Delta t \cdot \sqrt{1 - \left(\frac{\Delta t_0}{\Delta t}\right)^2}.
\end{align*}

Вычисления:
\begin{align*}
S &= 3 \cdot 10^8 \cdot 20 \cdot 10^{-9} \cdot \sqrt{1 - \left(\frac{10 \cdot 10^{-9}}{20 \cdot 10^{-9}}\right)^2} \\
  &= 6 \cdot \sqrt{1 - 0.25} \ \text{(м)} \\
  &= 6 \cdot \sqrt{0.75} \ \text{(м)} \\
  &\approx 5.196 \ \text{м}.
\end{align*}


ОТВЕТ: \\
\(S \approx 5.196 \ \text{м}.\)